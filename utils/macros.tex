\newcommand{\citationNeeded}{\textcolor{magenta}{\textbf{[CITATION NEEDED!]}}\xspace}
\newcommand{\tableNeeded}{\textcolor{magenta}{\textbf{[TABLE NEEDED!]}}\xspace}
\newcommand{\figureNeeded}{\textcolor{magenta}{\textbf{[FIGURE NEEDED!]}}\xspace}
\newcommand{\highLight}[1]{\textcolor{magenta}{\MakeUppercase{#1}}}
\newcommand{\figtitle}[1]{\textbf{#1}}

\newcommand{\datasets}{data sets\xspace}
\newcommand{\dataset}{data set\xspace}

\newcommand{\editorialNote}[1]{\textcolor{red}{[\textit{#1}]}}
\newcommand{\ignore}[1]{}
\newcommand{\addTail}[1]{\textit{#1}.---}
\newcommand{\super}[1]{\ensuremath{^{\textrm{#1}}}}
\newcommand{\sub}[1]{\ensuremath{_{\textrm{#1}}}}
\newcommand{\dC}{\ensuremath{^\circ{\textrm{C}}}}
\newcommand{\tb}{\hspace{2em}}
\newcommand{\tn}{\tabularnewline}
\newcommand{\spp}[1]{\textit{#1}}

\providecommand{\e}[1]{\ensuremath{\times 10^{#1}}}

\newcommand{\change}[2]{{\color{red} #2}\xspace}
\newcommand{\thought}[1]{\textcolor{purple}{THOUGHT: #1}}

\newcommand{\widthFigure}[6]{\begin{figure}[htbp]
\begin{center}
    \includegraphics[width=#1\textwidth]{#2}
    \captionsetup{#3}
    \caption[#4]{#5}
    \label{#6}
    \end{center}
    \end{figure}}

\newcommand{\heightFigure}[6]{\begin{figure}[htbp]
\begin{center}
    \includegraphics[height=#1\textheight]{#2}
    \captionsetup{#3}
    \caption[#4]{#5}
    \label{#6}
    \end{center}
    \end{figure}}

\newcommand{\smartFigure}[5]{%
    \begin{figure}[htbp]
        \begin{center}
            \includegraphics[width=\textwidth,height=0.95\textheight,keepaspectratio]{#1}
            \captionsetup{#2}
            \caption[#3]{#4}
            \label{#5}
        \end{center}
    \end{figure}
}

\newcommand{\mFigure}[4]{\smartFigure{#1}{listformat=figList}{#2}{#3}{#4}\clearpage}
\newcommand{\embedHeightFigure}[5]{\heightFigure{#1}{#2}{listformat=figList}{#3}{#4}{#5}}
\newcommand{\embedWidthFigure}[5]{\widthFigure{#1}{#2}{listformat=figList}{#3}{#4}{#5}}
\newcommand{\siFigure}[4]{\smartFigure{#1}{name=Figure S, labelformat=noSpace, listformat=sFigList}{#2}{#3}{#4}\clearpage}


%% macro to make long strings breakable over lines
\makeatletter
\def\breakable#1{\xHyphen@te#1$\unskip}
\def\xHyphen@te{\@ifnextchar${\@gobble}{\sw@p{\allowbreak{}\xHyphen@te}}}
% \def\xHyphen@te{\@ifnextchar${\@gobble}{\sw@p{\hskip 0pt plus 1pt\xHyphen@te}}}
\def\sw@p#1#2{#2#1}
\makeatother


\newcommand{\makeSampleTable}[3]{
\begin{longtable}{llll}
  \caption{#1} \label{tab:#2} \\
  
  \hline 
  \multicolumn{1}{c}{Sample ID} & 
  \multicolumn{1}{c}{Species} & 
  \multicolumn{1}{c}{Latitude} & 
  \multicolumn{1}{c}{Longitude} \\ 
  \hline 
  \endfirsthead
  
  \multicolumn{4}{c}%
  {{\tablename\ \thetable{} -- continued from previous page}} \\
  
  \hline 
  \multicolumn{1}{c}{Sample ID} & 
  \multicolumn{1}{c} {Species} & 
  \multicolumn{1}{c} {Latitude} & 
  \multicolumn{1}{c} {Longitude} \\ 
  \hline 
  \endhead
  
  \hline \multicolumn{4}{r}{{Continued on next page}} \\
  \endfoot
  
  \hline 
  % \hline
  \endlastfoot
  
  \input{#3}

\end{longtable}
\clearpage

}