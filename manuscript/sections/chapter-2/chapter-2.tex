\section{Introduction}
\ldots
\section{Methods}
\subsection{Sampling and DNA Isolation}
I collected genetic samples from \textit{A. americanus} and \textit{A. terrestris}
by driving roads during rainy nights between 2017 and 2020
in an a region of central Alabama where hybridization has previously been
inferred from the presence of morphological intermediates \parencite{weatherby1982}. 
I euthanized individuals with immersion in MS-222.
I removed liver and/or toes and preserved them in 100\% ethanol.
Samples and formalin fixed specimens were deposited in the Auburn Museum of Natural History.
Additional samples were also provided by museums.
I isolated DNA by first lysing a small piece of liver or toe approximately 
the size of a grain of rice in 300 \uL\ of a solution of 10mM Tris-HCL, 10mM EDTA, 
1\% SDS (w/v), and nuclease free water along with 6 mg Proteinase K that was 
incubated for 4-16 hours at 55\degree C in a 1.5 mL microcentrifuge tube.  
To purify the DNA and separate it from the lysis product, I mixed the lysis 
product with a 2X volume of SPRI bead solution containing 1 mM EDTA,  
10 mM Tris-HCl, 1 M NaCl, 0.275\% Tween-20 (v/v), 18\% PEG 8000 (w/v), 
2\% Sera-Mag SpeedBeads (GE Healthcare PN 65152105050250) (v/v), and nuclease free water.
I then incubated the samples at room tempearture for 5 minutes, placed the 
beads on a magnetic rack, and discarded the supernatant once the beads had collected
on the side of the tube.  
I then performed two ethanol washes by adding 1 mL of 70\% ETOH to the beads
while still placed in the magnet stand and allowing it to stand for 5 minutes
before discarding the ethanol. 
After removing all ethanol from the second wash, I removed the tube from the magnet 
stand and allowed the sample to dry for 1 minute before mixing the beads with 100 \uL\ of 
TLE solution containing 10 mM Tris-HCL, 0.1 mm EDTA, and nuclease free water.
After allowing the bead mixture to stand at room temperature for 5 minutes I returned
the beads to the magnet stand, pipetted all of the TLE solution into another 
microcentrifuge tube, and discarded the beads. I quantified DNA with a Qubit
flurometer (Life Technologies, USA) and diluted samples with TLE solution to 
bring the concentration to 20 ng/\uL.

\subsection{RADseq Library Preparation and Sequencing}
I prepared RADseq libraries using the 2RAD approach outlined by \cite{Bayona-vasquez2019}. 
On 96 well plates I ligated 100 ng of sample DNA in 15 \uL\ of a solution with 
1X Cutsmart Buffer (New Englad Biolabs, USA; NEB), 10 units of XbaI,
10 units of EcoRI, 0.33 \uM\ XbaI compatible adapter, 0.33 \uM\ EcoRI compatible adapter,
and nuclease free water with a 1 hour incubation at 37\degree C. 
I then immediately added 5 \uL\ of a solution with 1X Ligase Buffer (NEB),
0.75 mM ATP (NEB), 100 units DNA Ligase (NEB), and nuclease free water 
and incubated at 22\degree C for 20 min and 37\degree C for 10 min for two cycles, 
followed by 80\degree C for 20 min to stop enzyme activity.
For each 96 well plate, I pooled 10 \uL\ of each sample and split this pool 
equally between two microcentrifuge tubes.
I purified each pool of libraries with a 1X volume of SpeedBead solution followed 
by two ethanol washes as described in the previous section except that the DNA 
was resuspended in 25 \uL\ of TLE solution. 

% In order to be able to detect and remove PCR duplicates, I performed a single   
I performed a single   
cycle of PCR with the iTru5-8N primer which adds a random 8 nucleotide barcode to 
each library construct.  
For each plate I prepared four PCR reactions with a total volume of 
50 \uL\ containing 1X Kapa Hifi Buffer (Kapa Biosystems, USA; Kapa),
0.3 \uM\ iTru5-8N Primer, 0.3 mM dNTP, 1 unit Kapa HiFi DNA Polymerase,
10 \uL\ of purified ligation product, and nuclease free water.
I ran reactions through a single cycle of PCR on a thermocycler at 98\degree C for 2 min, 
60\degree C for 30 s, and 72\degree C for 5 min. 
I pooled all of the PCR products for a plate into a single tube and purified the
libraries with a 2X volume of SpeadBead solution as described before and 
resuspended in 25 \uL\ TLE.
I added the remaining adapter and index sequences unique to each plate with four PCR
reactions with a total volume of 50 \uL\ containing 1X Kapa Hifi (Kapa),
0.3 \uM\ iTru7 Primer, 0.3 \uM\ P5 Primer, 0.3 mM dNTP, 1 unit of Kapa Hifi DNA Polymerase (Kapa),
10 \uL\ purified iTru5-8N PCR product, and nuclease free water.
I ran reactions on a thermocycler with an initial denaturation at 98\degree C for 2 min, 
followed by 6 cycles of 98\degree C for 20 s, 60\degree C for 15 s, 72\degree C 
for 30 s and a final extension of 72\degree C for 5 min.
I pooled all of the PCR products for a plate into a single tube and purified the
product with a 2X volume of SpeadBead solution as described before and 
resuspended in 45 \uL\ TLE.

I size selected the library DNA from each plate in the range of 450-650 base pairs using
a BluePippin (Sage Science, USA) with a 1.5\% dye free gel with internal R2 standards. 
To increase the final DNA concentrations I prepared four PCR reactions for each 
plate with 1X Kapa Hifi (Kapa), 0.3 \uM\ P5 Primer, 0.3 \uM\ P7 Primer, 0.3 mM dNTP, 
1 unit of Kapa HiFi DNA Polymerase (Kapa), 10 \uL\ size selected DNA, and 
nuclease free water and used the same thermocycling conditions as the previous
(P5-iTru7) amplification.
I pooled all of the PCR products for a plate into a single tube and purified 
the product with a 2X volume of SpeadBead solution as before and resuspended in 20 \uL\ TLE. 
I quantified the DNA concentration for each plate with a Qubit fluorometer 
(Life Technologies, USA) then pooled each plate in equimolar amounts relative 
to the number of samples on the plate and diluted the pooled DNA to 5 nM with
TLE solution. 
The pooled libraries were pooled with other projects and sequenced on an Illumina 
HiSeqX by Novogene (China) to obtain paired end, 150 base pair sequences. 

\subsection{Data Processing}

I demultiplexed plate indexes using the process_radtags command from Stacks2 v2.6.1 \parencite{catchen2013},
allowing for 2 mismatches for rescuing reads. 











\subsection{Population Structure \& Identification Of Hybrids}
\ldots


\section{Results}
\subsection{Data Processing}
