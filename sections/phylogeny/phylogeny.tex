% Hybridization has become increasingly recognized to be a widespread
% phenomenon among closely related species. 
% However, it is not yet clear how important the consequences are.
% Potential consequences are adaptive introgression, introgression of neutral 
% genetic variation, reinforcement, lineage fusion, polyploidization, hybrid 
% speciation, transition to unisexual reproduction \parencite{abbott2013}. 

% The frequency of these outcomes is unknown.
% Assessing the impact of hybridization in the past is very difficult.
% Its effects become obscured with time and are not easily teased apart from other processes.
% It is much more difficult to infer models of evolution that include reticulations 
% than without due to the substantially greater number of possibilities that need 
% to be assessed.

% The degree of gene exchange that occurs in nature has been a highly debated topic. 
% Recent genomic studies have demonstrated that introgression is prevalent among 
% various groups of organisms, including fungi, vertebrates, insects, and angiosperms. 
% The consequences of introgression are diverse and depend on ecological and 
% genomic factors. 
% While it was previously believed to be mainly harmful, it is now evident that 
% introgression can be a source of genetic variation utilized in local adaptation 
% and adaptive radiation. Although our understanding of introgression as a 
% widespread phenomenon has improved, it remains uncertain how frequently it 
% occurs across different groups of organisms. 
% Ideally, determining the frequency of introgression throughout the tree of life 
% would involve analyzing clade-level genomic data systematically, without prior 
% knowledge of taxa known to hybridize in nature.

% Hybridization is a common phenomenon in nature and one which can have important 
% implications for evolution (Mallet, 2005). The role of hybridization in evolution 
% is becoming increasingly appreciated in a wide array of organisms and can have 
% varied effects on populations involved (A Genomic Perspective on Hybridization 
% and Speciation, 2016). Hybridization often results in reduced fecundity or the 
% production of unfit or sterile offspring when populations have sufficiently 
% diverged from one another. This cost has the potential to drive the evolution of 
% greater divergence between species when selection acts on traits that enhance 
% pre-mating barriers and reduce the likelihood hybridization (Servedio, 2001). 
% However, hybridization can also serve as a source of novel, adaptive genetic variation 
% through introduction of novel genetic variants or combinations thereof through 
% introgression between populations (Oziolor et al., 2019; Pardo-Diaz et al., 2012). 
% It has even been suggested that selection could maintain some degree of compatibility
% between species as a result of occasional introgression of adaptive genetic 
% variation (Servedio & Hermisson, 2020). In rare instances hybridization may 
% result in the emergence of new species from hybrid populations (Schumer et al., 2014). 
% If barriers to hybridization are not sufficiently strong, hybridization could 
% lead to the erosion of barriers between populations (Chafin et al., 2019). 
% Given that it can have important consequences in the evolution of lineages, 
% inference of historic introgression is important for a complete understanding of 
% a lineage’s evolutionary history and for understanding the consequences of 
% contemporary hybridization. 

% North American toads in the genus Anaxyrus have been the subject of many studies 
% of contemporary hybridization but we know very little about the historical context 
% preceding it. Many species pairs have been found to hybridize naturally or in the 
% laboratory (Blair, 1972; Green, 1996). Several attributes make toads appealing 
% for the study of hybridization. They are very abundant and conspicuous, and their 
% primary sexual signal – advertisement call— can be readily measured and quantified. 
% There has been success with rearing them in the lab and they produce very large clutch sizes. 
% Sex determination in toads is unique among species with historical prominence in
% hybridization research. Many bufonid toads are known to possess homomorphic sex chromosomes. 
% There have also been several transitions between XY and ZW sex determination. 
% This is interesting because many studies have found evidence for an important 
% role for sex chromosome in the evolution of reproductive incompatibility (Coyne & Orr, 2004). 
% But most studies have focused on species with heteromorphic sex chromosomes 
% where hybridization occurs between species with XY sex determination systems.

% While North American toads are an attractive system for gaining more understanding 
% of the process of speciation and evolution of reproductive incompatibility, 
% we currently lack important historical context for hybridization in these toads. 
% Foremost is uncertainty about the evolutionary relationships among species and populations. 
% It is not clear how closely related hybridizing species are to one another and whether 
% they are each other’s closest living relatives. A number of studies have estimated their 
% phylogenetic relationships, but there is some disagreement among these studies 
% (Fontenot et al., 2011; Graybeal, 1997; Masta et al., 2002; Pramuk et al., 2008; Pyron et al., 2011). 
% Particularly in the americanus group which is composed of 7 currently recognized species. 
% The inconsistencies in inferred relationships could be due to many possible 
% sources of error known to affect phylogenetic inference (Som, 2014). 
% Furthermore, some studies only included a single sample from currently 
% recognized species, and thus were not able to confirm the monophyly of these species 
% (Pyron & Wiens, 2011). In studies that did sample multiple individuals per
% species, some, like A. americanus and A. fowleri, were inferred to be polyphyletic (Fontenot et al., 2011).
% The inconsistencies in inferred relationships among Anaxyrus species could be caused by 
% real biological processes such as incomplete lineage sorting or introgressive 
% hybridization (Degnan & Rosenberg, 2009; Kubatko & Degnan, 2007).
% Incomplete lineage sorting or hybridization could result in individual gene 
% trees that differ from the true species history. Graybeal, 1997 and Masta et al., 2002 used only mitochondrial sequence data which may not be consistent with the evolutionary history of the nuclear genome or that of the species due to incomplete lineage sorting or mitochondrial introgression. Fontenot et al., 2011, Pramuk et al., 2008, and Pyron & Wiens, 2011 used concatenated alignments of mtDNA and a small number of nuclear loci. However, concatenated analyses can be misleading when there is high discordance among gene trees due to incomplete lineage sorting (Kubatko & Degnan, 2007; Roch & Steel, 2015). Given the large number of putative hybrids observed between contemporary populations of toads, historic hybridization resulting in a significant amount of gene tree discordance seems highly plausible. Quantifying how much discordance is caused by hybridization would also be important context for understanding contemporary hybridization and the evolution of reproductive barriers between species. A history of hybridization and admixture among ancestral lineages could play an important role in the degree of reproductive isolation that exists between species of toads today. 
% Knowing when species have diverged is also important context for understanding 
% contemporary hybridization and the evolution of reproductive isolation. 
% It is still unknown how time since divergence factors into the evolution of 
% reproductive incompatibility. Knowledge of divergence times can also shed light 
% on drivers of speciation. For example, divergence times that are shared among 
% different pairs of species are particularly interesting in that they suggest the 
% processes that have driven diversification may have been shared among taxa.

% I propose to obtain genome wide sequence data from samples representing a 
% large portion of the ranges of most described species in the genus Anaxyrus. 
% With these data, I will test the following hypotheses: 1) recent biogeographic 
% processes have reduced or eliminated gene flow between populations within species; 
% 2) contemporary hybridization occurs between species with overlapping or abutting ranges; 
% 3) ancient introgression has occurred between lineages; 
% 4) large-scale biogeographic processes have driven diversification in Anaxyrus; 
% and 5) the current taxonomy of Anaxyrus is not consistent with evolutionary 
% relationships among populations within named groups. 

\section{Introduction}
% Many factors influence diversification.
% - Environment is really important
%   - Changes connectivity
%     - Creates opportunity for divergence
%       - Adaptive or neutral
%   - Drives adaptive change
%   - Influences population size which in turn affects drift
% - Degree of isolation is influenced by environment
% - Indirectly related to environment are the effects of introgression.
%   - Enabled by environmental changes that effect connnectivity.
%   - Itself can have important effects on evolution.
% - Understanding the interplay of all of these factors is critical for understanding
% diversification of organisms.

Many factors are hypothesized to be important in driving and shaping the 
diversification and evolutionary history of organisms. 
Chief among them is the interplay between climatic conditions and geologic processes
\parencite{hua2013}.
Changes in these environmental variables can alter the distributions of organisms
or alter the distribution of variation within species and as a result,
change patterns of gene flow within a species \parencite{coyne2004}.
Spatially separated populations may undergo genetic divergence from one another due 
to adaptive evolution in response to changing abiotic or biotic conditions 
or they might simply diverge via neutral evolution driven by the effects of drift \parencite{coyne2004}.
Local adaptation in response to the environment may lead assortative mating 
among populations of a species which in time could result in complete reproductive
isolation \parencite{mallet2008}.  
Another important process, which itself will often be tied to environmental changes
is hybridization. 
Environmental changes can reestablish migration between previously isolated 
populations resulting in hybridization and potentially introgression between species 
\parencite{abbot2013}.
Understanding the interplay of all of these factors is critical for understanding
the evolutionary history of organisms.
A critical step to understanding these processes is obtaining an accurate 
reconstruction of the evolutionary history of organisms. 
Knowing the order of divergences along with their timing, it may be possible to
identify the underlying environmental factors driving them. 
Revealing signatures of introgression in the past paves the way for  
understanding it's consequences and role in the diversification process. 


%%%%%%%%%%%%%%%%%%%%%%%%%%%%%%%%%%%%%%%%%%%%%%%%%%%%%%%%%%%%%%%%%%%%%%%%%%%%%%%%
% Introducing anaxyrus
The North American toads in the genus \anaxyrus are a group of organisms with a 
poorly understood evolutionary history. 
Although, not for lack of trying.
Multiple studies of the evolutionary relationships among species in the genus have produced
conflicting results \parencite{fontenot2011,graybeal1997,masta2002,pramuk2007,pyron2011,portik2023}.
Particularly within the \americanus group composed of \amer, \baxteri, \fowl, 
\hemiophrys, \houstonensis, \terr, and \wood.
Two phylogenetic studies have inferred trees with \fowl forming a polytomy making 
them inconsistent with the current taxonomy of \anaxyrus \parencite{fontenot2011,masta2002}.
The conflicting results produced by these studies could be due to methodological differences such as
the species included, the number of individuals of each species sequenced, inference 
methods used, or the sequenced loci. 
But the differences in inferred relationships could also result from real biological processes. 
Incomplete lineage sorting is one potential source of discordance among datasets 
which include different loci that arises from real biological processes and 
impacts phylogenetic inference \parencite{kubatko2007}. 
Incomplete lineage sorting could also produce the polytypic relationship among
\fowl. 

% Move to another paragraph towards the end
% Only two studies have specifically examined patterns of genetic diversity within
% species and included multiple samples per species \parencite{fontenot2011,masta2003}.
% \cite{fontenot2011} using AFLPs found limited evidence...
% \cite{masta2003} found two distinct clades of \wood
% Found polyphyly with \fowl.
% Several subspecies have been recognized in the past raising the possibility of
% significant differentiation within species or possibly even unrecognized diversity.
% A. americanus charlsmithi
% A. americanus americanus
% A. americanus copei
% A. woodhousii velatus
% A. woodhousii australis
% A. woodhousii woodhousii




%%%%%%%%%%%%%%%%%%%%%%%%%%%%%%%%%%%%%%%%%%%%%%%%%%%%%%%%%%%%%%%%%%%%%%%%%%%%%%%%
% Potential for introgression in anaxyrus
Gene flow is another potential source of discordance among genes which could  
drive the differences in inferred relationships among studies using different 
loci and could also produce the pattern seen in \fowl \parencite{degnan2009}. 
While incomplete lineage sorting is very likely to have impacted patterns of genetic  
variation in \anaxyrus, gene flow due to hybridization is a distinct possibility as well.
There are numerous reports of natural hybridization between several different 
species of \anaxyrus \parencite{green1996}. 
A study of allozyme variation across a hybrid zone between \amer and 
\hemiophrys revealed introgression taking place across a more than 50km wide hybrid zone.
In the previous chapter I presented the results of a study on the hybrid zone
between \amer and \terr \parencite{green1983}.
\cite{meacham1962} presented compelling evidence on the basis of morphological 
variation for the existence of a hybrid zone between \fowl and \wood in East Texas
although this has never been investigated with genetic data. 
In the previous chapter, I presented evidence for extensive hybridization between
\amer and \terr.
Furthermore, numerous laboratory crosses have been performed between pairs of \anaxyrus 
species that occur in sympatry \parencite{blair1972,blair1963}.
Some of which produce viable and fertile backcross progeny \parencite{blair1972,blair1963}.
These studies suggest that gene flow could very well have played a role in shaping patterns
of diversity in \anaxyrus. 
However, these studies provide only a snapshot in time with no indication of the long term 
evolutionary consequences if any.
There are many potential lasting consequences of hybridization such as adaptive introgression, 
introgression of neutral genetic variation, reinforcement, lineage fusion, 
polyploidization, hybrid speciation, or transition to unisexual reproduction \parencite{abbott2013}. 
Inference of past introgression is an important starting point for exploring these
outcomes yet it remains a challenging problem. 
Inferring the structure of phylogenetic networks is much more computationally 
demanding than inferring more simple bifurcating phylogenetic trees \parencite{wen2018}. 
There has been some recent work to overcome this challenge as well as increased 
feasibility of obtaining appropriate genome wide datasets to investigate past 
gene flow making.


%%%%%%%%%%%%%%%%%%%%%%%%%%%%%%%%%%%%%%%%%%%%%%%%%%%%%%%%%%%%%%%%%%%%%%%%%%%%%%%%
% Hybridization is useful but needs context
Apart from the significant evolutionary implications of hybridization which 
need to be understood, it also presents a valuable opportunity for investigating 
the mechanisms that drive divergence and the evolution of reproductive
incompatibility \parencite{rieseberg1999}. 
Many generations of backcrossing within hybrid zones can produce a large number
of highly recombinant genomes that allow for the observation of many possible 
hybrid genotypes under natural conditions in order to identify advantageous or
disadvantageous hybrid genotypes \parencite{rieseberg1999}. 
In most species it is not feasible to produce such a large number of highly  
recombinant offspring in order to make such observations.
The evolutionary history of hybridizing species is important context to have 
when studying hybrid zones. 
Context such as the phylogenetic relationships of hybridizing species, 
the amount of genetic divergence between them,  
the time since divergence, and the biogeographic 
processes driving initial divergence and subsequent secondary contact in cases 
of allopatric divergence.
This important context is currently missing for \anaxyrus which limits our
understanding of hybridization within the genus.


%%%%%%%%%%%%%%%%%%%%%%%%%%%%%%%%%%%%%%%%%%%%%%%%%%%%%%%%%%%%%%%%%%%%%%%%%%%%%%%%
% Effect of environment on toads
Ultimately, changes in the environment are what drive speciation and hybridization
and so it is important to identify these.
To date, there have not been any studies conducted to understand how the 
environment as driven diversification in North American toads. 
North America has had a very complex geologic and climatic history \parencite{lyman2022}.
The effects of which are often clade specific \parencite{nunez2023}.
But large scale environmental changes can impact multiple species simultaneously \parencite{oaks2019,xue2015}.
There has been recent development in methods to infer these events \parencite{oaks2019,oaks2022}.
The identification of multiple pairs of lineages that underwent divergence at 
simultaneously can provide clues as to the cause.  
Present day population structure could also provide further understanding by 
revealing environmental factors that reduce gene flow assuming the biological 
limits of present day species have not evolved dramatically from their ancestral 
state.


%%%%%%%%%%%%%%%%%%%%%%%%%%%%%%%%%%%%%%%%%%%%%%%%%%%%%%%%%%%%%%%%%%%%%%%%%%%%%%%%
% Goals of this study
In this study, I investigate the evolutionary history of North American
toads in the genus \anaxyrus using genome wide sequence data. 
For this I obtained restriction enzyme-associated DNA sequence (RADseq) data
from 12 species of \anaxyrus, including sampling that encompasses a 
large portion of the ranges of \amer, \fowl, \terr, and \wood.
With these data I conduct the first inference the evolutionary relationships 
among these species using genome wide sequence data. 
I also test for the presence of shared divergence times which might suggest
\anaxyrus diversification has been driven by the same environmental changes
and also estimate the absolute timing of all divergences within the genus.
With the robust estimate of phylogenetic relationships among \anaxyrus species, 
I test for the presence of ongoing and historic introgression among \anaxyrus species.
In order to identify the types of environmental factors that might have played
a role in isolating populations that would eventually diverge as species,
I investigate population structure within a subset of \anaxyrus species.
Finally, I estimate proportions admixture between \fowl and \wood to test
the hypothesis that these species form a hybrid zone in the central United  
States where their ranges meet.

% 1) recent biogeographic 
% processes have reduced or eliminated gene flow between populations within species; 
% 2) contemporary hybridization occurs between species with overlapping or abutting ranges; 
% 3) ancient introgression has occurred between lineages; 
% 4) large-scale biogeographic processes have driven diversification in Anaxyrus; 
% and 5) the current taxonomy of Anaxyrus is not consistent with evolutionary 
% relationships among populations within named groups. 


%%%%%%%%%%%%%%%%%%%%%%%%%%%%%%%%%%%%%%%%%%%%%%%%%%%%%%%%%%%%%%%%%%%%%%%%%%%%%%%%
\section{Methods}
\subsection{Sampling and DNA Isolation}
% TODO: How many samples
I obtained tissue samples from museum tissue collections as well as individuals   
that I collected from 2017 to 2020. I selected samples to represent as much
of the range of each species of \textit{Anaxyrus} as possible.
I also included one \textit{incilius nebulifer} as an outgroup for phylogenetic analyses.

I isolated sample DNA from liver or muscle tissue by lysing a piece approximately 
the size of a grain of rice in a 300 \uL\ solution of 10mM Tris-HCL, 10mM EDTA, 
1\% SDS (w/v), 6 mg Proteinase K, and nuclease free water incubated for 4-12 
hours at 55\degree C.  
To purify the DNA, I mixed the lysis solution with a 2X volume of SPRI bead 
solution containing 1 mM EDTA, 10 mM Tris-HCl, 1 M NaCl, 0.275\% Tween-20 (v/v), 18\% PEG 8000 (w/v), 
2\% Sera-Mag SpeedBeads (GE Healthcare PN 65152105050250) (v/v), and nuclease free water.
I then incubated the samples at room temperature for 5 minutes, placed the 
beads on a magnetic rack, and discarded the supernatant after beads had collected
on the side of the tube.  
I then performed two ethanol washes with 1 mL of 70\% ETOH added to the beads
while still placed in the magnet stand and allowed the sample to stand for 5 minutes
before discarding the ethanol. 
After discarding all ethanol from the second wash, I removed the tube from the magnet 
stand and allowed the sample to dry for 1 minute.
I then mixed the beads with 100 \uL\ of TLE solution containing 10 mM Tris-HCL, 
0.1 mm EDTA, and nuclease free water.
After allowing this mixture to stand at room temperature for 5 minutes I returned
the beads to the magnet stand and separated the DNA solution from the beads.
I quantified DNA with a Qubit fluorometer (Life Technologies, USA) and diluted 
samples with TLE solution to bring all sample concentrations to 20 ng/\uL.

\subsection{RADseq Library Preparation}
I prepared RADseq libraries using the 2RAD approach outlined by \cite{bayona-vasquez2019}. 
On 96 well plates, I digested 100 ng of sample DNA in 15 \uL\ of a solution with 
1X CutSmart Buffer (New England Biolabs, USA; NEB), 10 units of XbaI,
10 units of EcoRI, 0.33 \uM\ XbaI compatible adapter, 0.33 \uM\ EcoRI compatible adapter,
and nuclease free water with a 1 hour incubation at 37\degree C. 
I ligated the adapter by adding 5 \uL\ of a solution with 1X Ligase Buffer (NEB),
0.75 mM ATP (NEB), 100 units DNA Ligase (NEB), and nuclease free water 
and incubated at 22\degree C for 20 min and 37\degree C for 10 min for two cycles, 
followed by 80\degree C for 20 min to stop enzyme activity.
For each 96 well plate, I pooled 10 \uL\ of each sample and split this pool 
into equal volumes.
I purified each pool of libraries with a 1X volume of SPRI bead solution followed 
by two ethanol washes as described in the previous section except that the DNA 
was resuspended in 25 \uL\ of TLE solution. 

In order to be able to detect and remove PCR duplicates, I performed a single   
cycle of PCR with the iTru5-8N primer which adds a random 8 nucleotide barcode to 
each library construct.  
For each plate, I prepared four PCR reactions with a total volume of 
50 \uL\ containing 1X Kapa Hifi Buffer (Kapa Biosystems, USA; Kapa),
0.3 \uM\ iTru5-8N Primer, 0.3 mM dNTP, 1 unit Kapa HiFi DNA Polymerase,
10 \uL\ of purified ligation product, and nuclease free water.
I ran reactions through a single cycle of PCR on a thermocycler at 98\degree C for 2 min, 
60\degree C for 30 s, and 72\degree C for 5 min. 
I pooled all of the PCR products for a plate into a single tube and purified the
libraries with a 2X volume of SpeedBead solution as described before and 
resuspended in 25 \uL\ TLE.
I added the remaining adapter and index sequences unique to each plate with four PCR
reactions with a total volume of 50 \uL\ containing 1X Kapa Hifi (Kapa),
0.3 \uM\ iTru7 Primer, 0.3 \uM\ P5 Primer, 0.3 mM dNTP, 1 unit of Kapa Hifi DNA Polymerase (Kapa),
10 \uL\ purified iTru5-8N PCR product, and nuclease free water.
I ran reactions on a thermocycler with an initial denaturation at 98\degree C for 2 min, 
followed by 6 cycles of 98\degree C for 20 s, 60\degree C for 15 s, 72\degree C 
for 30 s and a final extension of 72\degree C for 5 min.
I pooled all of the PCR products for a plate into a single tube and purified the
product with a 2X volume of SpeedBead solution as described before and 
resuspended in 45 \uL\ TLE.

I size selected the library DNA from each plate in the range of 450-650 base pairs using
a BluePippin (Sage Science, USA) with a 1.5\% dye free gel with internal R2 standards. 
To increase the final DNA concentrations I prepared four PCR reactions for each 
plate with 1X Kapa Hifi (Kapa), 0.3 \uM\ P5 Primer, 0.3 \uM\ P7 Primer, 0.3 mM dNTP, 
1 unit of Kapa HiFi DNA Polymerase (Kapa), 10 \uL\ size selected DNA, and 
nuclease free water and used the same thermocycling conditions as the previous
(P5-iTru7) amplification.
I pooled all of the PCR products for a plate into a single volume and purified 
the product with a 2X SPRI bead solution as before and resuspended in 20 \uL\ TLE. 
I quantified the DNA concentration for each plate with a Qubit fluorometer 
(Life Technologies, USA) then pooled each plate in equimolar amounts relative 
to the number of samples on the plate and diluted the pooled DNA to 5 nM with
TLE solution. 
These pooled libraries were pooled with other projects and sequenced on an Illumina 
HiSeqX by Novogene (China) to obtain paired end, 150 base pair sequences. 

\subsection{Phylogenetic Data Processing}
To produce alignments for phylogenetic analysis, I first I demultiplexed the 
iTru7 indexes (identifying the 96 well plates) using the \processradtags command from \stacks v2.6.4 
\parencites{rochette2019} and allowed for two mismatches for rescuing reads.
I removed PCR duplicates using the the \clonefilter command from \stacks.
To demultiplex individual samples I used \pyrad v0.9.90 and allowed for one 
mismatch for rescuing reads. 
I assembled and aligned reads with \pyrad using default parameters and a 
clustering threshold of 0.8. 
Using \pyrad, I filtered loci not present in at least 75\% of samples 
and filtered samples with fewer than 200 loci.

\subsection{Maximum Likelihood}
Phylogenetic methods that do not account for incomplete lineage sorting  
do not perform well with data impacted by this process.
However, methods that do account for incomplete lineage sorting are far more 
computationally demanding.
As a result, these methods cannot be performed with a large number of samples.
I therefore conducted conducted maximum likelihood phylogenetic inference in  
order to infer a phylogeny with all of the sequenced samples in order  
to identify samples that may be problematic for other methods due to recent 
admixture, data quality, or misidentification. 
I conducted the maximum likelihood phylogenetic inference with \iqtree 
v1.6.12 \parencite{nguyen2015} with the \pyrad alignment as input. 
I ran \iqtree with 1000 ultrafast bootstrap replicates \parencite{hoang2018}
under the GTR substitution model.

\subsection{Multispecies Coalescent}
In order to account for incomplete lineage sorting in the inference of phylogenetic
relationships and to infer shared divergence times, I used the program \phycoeval \parencite{oaks2022}.
I selected a subset of up to four samples from each species due to
infeasible run times for \phycoeval with greater numbers of samples (see \cref{table:phylo}).
I excluded sample 006 from consideration due it having an anomalous   
position in the maximum likelihood tree.
I used \pyrad to filter loci not present in at least 75\% of samples. 
Using a custom script, I filtered the phylip alignment file produced by \pyrad 
to exclude sites with more than two characters and output the filtered alignment 
in the nexus format with a biallelic character encoding. 
I ran \phycoeval with state frequencies fixed at 0.5.
I set the mutation rate equal to one so that divergence times are in units of 
expected substitutions per site. 
I set the prior on the age of the root as an exponential distribution with a mean
of 0.01.
I ran \phycoeval with the assumption of a single effective population size
shared across all of the branches of the tree.
The prior on the effective population size was a gamma prior with a shape of
four and mean of 0.0005
I ran five independent MCMC chains for 10,000 generations, sampling every 10 
generations.
Each chain was started with a comb tree topology with all branches sharing the
same divergence time. 
I summarized the posterior sample of tree topologies and parameters using 
\sumphycoeval which is packaged with \phycoeval \parencite{oaks2022}.
To assess convergence and mixing, I used \sumphycoeval to calculate the
potential scale reduction factor (PSRF) and the effective sample size (ESS).
I discarded the first 100 samples from each chain as burnin.
I used \sumphycoeval to rescale the branch lengths of the maximum a posteriori 
(MAP) tree produced by \sumphycoeval so that the posterior mean root age was 
16.5 million years ago based on the estimate of \cite{feng2017}.

\subsection{Introgression}
In order to test for introgression between species of \anaxyrus, I used the 
program \dsuite v0.5r50 \parencite{malinsky2021} to compute the $f$-branch 
statistic for each pair of \anaxyrus species for which the statistic 
can be calculated \parencite{reich2009,malinsky2018}. 
I used \pyrad to filter all loci that were not found in at least 50\% of the 
samples that passed filtering and excluded sample 006 due to it's anomalous 
position in the maximum likelihood phylogeny.
For the input tree topology required to run \dsuite, I used the topology inferred
by \phycoeval and I specified \nebulifer as the outgroup species.
I ran the \dsuite Dtrios command to compute Patterson's the $f4$-ratio
statistic for all possible trios with 20 block-jackknife replicates.
I then ran the Fbranch command from \dsuite to compute the $f$-branch statistics 
from the computed $f4$-ratio statistics. 
I plotted the $f$-branch statistics with \dtools v0.1 which is packaged with
the \dsuite program \parencite{malinsky2021}. 

\subsection{Population Structure}
I processed reads differently for the investigation of population structure within
\amer, \fowl, \terr, and \wood as well as for the investigation of hybridization
between \fowl and \wood. 
Starting from the decloning step of the data processing for the phylogenetic 
analyses, I demultiplexed individual samples using the \processradtags program 
in \structure. 
I also trimmed adapter sequence and filtered reads with low quality scores as 
well as reads with any uncalled bases with \processradtags and allowed for the rescue of 
restriction site sequence as well as barcodes with up to two mismatches.  
I allowed for 14 mismatches between alleles within, as well as between individuals
(M and n parameters). This is equivalent to a sequence similarity threshold of   
90\% for the 140 bp length of reads post trimming. 
I also allowed for up to 7 gaps between alleles within and between individuals.
I used the \populations command from \stacks to filter loci missing in more than   
5\% of individuals, filter all sites with minor allele counts less than 3, filter 
any individuals with more than 90\% missing loci, and randomly sample a single
SNP from each locus.

I ran the program \structure v2.3.4 \parencite{pritchard2000} with its admixture 
model for each species separately and with \fowl and \wood samples combined 
in order to cluster individuals and estimate their ancestry proportions. 
For the \structure analyses of individuals from a single species, I ran \structure
under five different models with each assuming a different number of populations
(K parameter) ranging from one to five. 
For the \structure analysis of the combined \fowl and \wood samples, I ran 
\structure under four different models with K ranging from one to four.
I ran 10 independent runs of \structure for each value of K for a total of 100,000
steps and burnin of 50,000 for each run.
I used the R package \pophelper v2.3.1 \parencite{francis2017} to combine
runs for each value of K and to select the model resulting in the largest
$\Delta$K which is the the model that has the greatest increase in likelihood 
score from the previous model which assumed one less population as described by \parencite{evanno2005}.
I also investigated population structure with a non-parametric approach, using
principle component analysis (PCA) implemented in the R package \adegenet
\adegenet v2.1.10 \parencite{jombart2008}. 

%%%%%%%%%%%%%%%%%%%%%%%%%%%%%%%%%%%%%%%%%%%%%%%%%%%%%%%%%%%%%%%%%%%%%%%%%%%%%%%%
\section{Results}

\subsection{Assembly and alignment with \pyrad}
A total of 436,265,266 reads were obtained for all samples. After filtering low
quality reads and reads without restriction site sequence, 435,650,926 total reads 
remained for assembly.
The number of filtered reads per individual was highly variable with a mean of 
4,538,030 (sd=3,619,076).
Prior to filtering there were 171,174 loci total loci which was reduced to  
659 after filtering loci not present in at least 75\% of samples and filtering
samples which had fewer than 200 loci (see \cref{table:phylo}).
Mean sequence read coverage of the loci passing filter was 54x.
The final alignment contained a total of 184,453 sites and 20,361 SNPs with 
14.96\% of sites and 14.71\% of SNPs missing.

\subsection{Maximum Likelihood Phylogeny}
The full majority rule consensus tree inferred by \iqtree is presented in \iqtreefigs. 
All species were inferred as a single monophyletic group with the exception 
of \fowl. 
A single \fowl sample (sample 006) does not form a monophyletic group with other 
\fowl samples but is instead sister to the the branch containing \wood and \fowl samples (\iqtreefigs).
A representation of the tree inferred by \iqtree with the tips within
species specific clades collapsed is presented in (\cref{fig:iqtree-collapsed}). 
Each species specific clade for which there are at least two representatives
samples all have ultrafast bootstrap support values of 100\% (\cref{fig:iqtree-collapsed}).
All branches below the the level of the species specific clades have ultrafast 
bootstrap support values ranging from 70-100\% with the majority being 100\% (\cref{fig:iqtree-collapsed}).
The most basal internal branch of the tree, marking the split between most of \anaxyrus
and \punctatus along with the outgroup \nebulifer has an ultrafast bootstrap 
support value of 99\% (\cref{fig:iqtree-collapsed}).
The sister branch to to \terr, which contains the spurious \fowl sample (sample 006)
and the clade containing \fowl and \wood, has an ultrafast bootstrap support 
value of 96\% (\cref{fig:iqtree-collapsed}).
The lowest ultrafast bootstrap support value is found on the branch sister 
to the \cognatus/\speciosus clade with a value of only 70\% (\cref{fig:iqtree-collapsed}).

\subsection{Coalescent Phylogeny}
The maximum a posteriori (MAP) tree inferred under the multispecies coalescent
model using \phycoeval has a topology that differs from the maximum likelihood topology 
inferred by \iqtree (\cref{fig:phycoeval}).
The MAP tree produced by \phycoeval does not have any shared divergence times 
among any of the 10 internal nodes of the tree (\cref{fig:phycoeval}).
The frequency of topologies in the posterior sample that have 10 independent 
divergence times is 0.5.
The next most frequent topology in the posterior are topologies with a single 
shared divergence time and nine independent divergences occurring with a frequency
of 0.24.
One major difference between the maximum likelihood tree inferred by \iqtree and 
the MAP tree inferred by \phycoeval is that the MAP tree has one multifurcation \. 
This multifurcation happens at the ancestor of the \quercicus, \speciosus/\cognatus,
and \amer group lineages (\cref{fig:phycoeval}).
However, this node has a low posterior probability of only 0.51 (\cref{fig:phycoeval}). 
All other branches in the MAP tree have high posterior probabilities of 0.98 or more (\cref{fig:phycoeval}). 
Most divergence events within \anaxyrus have occurred in the past 3.5 million 
years and all diversification within the \amer group is less than 2.5   
million years old (\cref{fig:phycoeval}).

\subsection{Introgression}
I used the program \dsuite to compute the \fbranch statistic which is an  
estimate of excess allele sharing between species pairs that is not due to 
incomplete lineage sorting. 
I used the species tree topology produced by \phycoeval for estimating 
the \fbranch statistics. 
The \fbranch estimates for each species pair are presented with a heat map in 
figure \cref{fig:dsuite}.
Most \fbranch estimates produced by \dsuite were zero or very near zero.
Only 24 out of 112 \fbranch estimates were greater than 0 and just 11 of those were 
greater than 0.05 \cref{fig:dsuite}.
\amer and \wood had the largest number of estimates greater than
zero associated with them with nearly every pairwise comparison greater than 0
\cref{fig:dsuite}. 
The highest \fbranch statistic values are between \amer and two other 
species: \hemiophrys (0.24) and \baxteri (0.22) \cref{fig:dsuite}.
The values associated with \wood are appreciably lower with none exceeding 
0.1 \cref{fig:dsuite}.
The highest being between \amer and \wood with a value of 0.098 \cref{fig:dsuite}.
The \wood \fbranch values for \baxteri and \hemiophrys are 0.082 and 0.086 
respectively \cref{fig:dsuite}.
The \fbranch value between \wood and \microscaphus is 0.05.
Finally, the smallest \wood \fbranch values are in the tests with \cognatus and 
\speciosus at 0.023 and 0.029 respectively.

\subsection{Population Structure}
For the \structure analysis of each species and the analysis of the \fowl and \wood
samples combined, a visual inspection of the 10 independent \structure runs 
performed for each value
of K, shows that each independent run converged on a nearly identical result 
for all runs for a given K value 
(\cref{fig:structure-all-amer,fig:structure-all-fowl,fig:structure-all-terr,fig:structure-all-wood}).
For the \amer, \fowl, \wood, and \fowl + \wood analyses, the \structure model 
with the highest $\Delta$K was the model with a K of two.   
(\cref{fig:evanno-amer,fig:evanno-fowl,fig:evanno-wood,fig:evano-fowl-wood}). 
For the \terr analysis, the \structure model with the highest $\Delta$K was 
the model with a K of three (\cref{fig:evanno-terr}).

The \structure analysis with a K of two for \amer produced a western and 
eastern cluster of individuals with four admixed samples in the center of 
the species range (\cref{fig:structure-americanus2}).
There was a large increase in likelihood between the model with a K of two and the
model with a K of three although it was not large enough to be identified as 
the best model using the method described by \cite{evanno2005}.
Therefore, the \structure results for the model with a K of three are also presented 
(\cref{fig:structure-americanus}).
The analysis performed with a K of three shows the same East/West division 
but also shows a gradient from North to South in the eastern half of the \amer 
range (\cref{fig:structure-americanus}).
The PCA for \amer also shows three non-discrete groupings of individuals \amer 
samples which more closely matches the \structure analysis with a K of three.

The ancestry coefficients inferred in the \structure analysis for \terr fall 
into three categories. 
Individuals in the first category, which includes all but four individuals, 
have admixture proportions attributed to two different source populations (Population 1 and Population 2)  
with the majority of ancestry attributed to Population 1 (\cref{fig:structure-terrestris}).
The second category of individuals, which includes the two easternmost samples, have  
ancestry proportions attributed to Population 1 and a third population 
(Population 3) (\cref{fig:structure-terrestris}).
The third category, which includes the next two easternmost samples, has 
ancestry proportions attributed to all three.
These samples resemble the first category except that they have a small amount
of ancestry attributable to population 3.
The PCA result for \terr is fairly consistent with the \structure results with 
most individuals clustering tightly together and three samples forming another \cref{fig:structure-terrestris}.

The results of the \fowl or \wood \structure analysis and PCA do not show any obvious 
population structure or pattern in the distribution of genetic diversity 
\cref{fig:structure-fowleri,fig:structure-woodhousii}.
Two \wood samples have ancestry coefficients of 1.0 for a separate population
than the remaining samples \cref{fig:structure-fowleri}.
However, when analyzing the \fowl and \wood samples together, these same two samples 
have a high proportion of \fowl ancestry and are also located 
in the center of the two ranges \cref{fig:structure-fowleri-woodhousii}.
Several other samples in the combined \fowl and \wood analysis have mixed ancestry 
with a small proportion of \wood ancestry and these too are located in the center of the two ranges
\cref{fig:structure-fowleri-woodhousii}. 
Again, the PCA results are consistent with the \structure results. 
The PCA plot shows \wood samples clustered tightly, two samples right in the 
middle of principal component one which captures 42\% of variation in the data. 
The \fowl samples are also tightly clustered except for four samples which 
gravitate towards the center of principal component one \cref{fig:structure-fowleri-woodhousii}. 


%%%%%%%%%%%%%%%%%%%%%%%%%%%%%%%%%%%%%%%%%%%%%%%%%%%%%%%%%%%%%%%%%%%%%%%%%%%%%%%%
\section{Discussion}

\subsection{Phylogenetic relationships}
% Maximum likelihood tree
The maximum likelihood tree inferred by \iqtree \cref{fig:iqtree,fig:iqtree-collapsed} 
differs from trees inferred in previous studies of the relationships 
among \anaxyrus \parencite{fontenot2011,graybeal1997,masta2002,pramuk2007,pyron2011,portik2023}.
Even among these previous studies there has been a great deal of inconsistency in
the inferred relationships except in the position of a few taxa.
As in all previous studies, the maximum likelihood tree inferred in this 
study places \punctatus sister to all other \anaxyrus.
I also found the \americanus group to be monophyletic and sister to \microscaphus 
which is consistent with most previous studies.
Two previous studies have inferred trees which do not place \fowl samples
into a single monophyletic group \parencite{masta2002,fontenot2011}.
In this study, a single \fowl sample included in this study does not fall within a monophyletic 
group with the remaining \fowl samples but is instead sister to the clade 
containing all \fowl and \wood samples \cref{fig:iqtree-collapsed}.

All of these studies have included different species, individuals, and loci, and 
also used different methods for alignment and phylogenetic inference. 
These differences in study design could result in the observed topology differences. 
The choice of locus in particular has a high likelihood of being the cause of these differences. 
Due to incomplete lineage sorting, the true histories of each gene may in fact 
differ from one another and not reflect the history of the species \parencite{kingman1982}. 
The practice of concatenating multiple loci as all previous studies of
\anaxyrus evolutionary relationships have done, can produce erroneous trees  
with high statistical support \parencite{kubatko2007}.
Despite the inappropriateness of concatenated analysis with genome-wide data,
it was reassuring to find that all but one individual clustered with members
of it's own species in my analysis. 
In my experience, \anaxyrus can be challenging to identify. 
Particularly in a preserved state. 
The maximum likelihood tree does not suggest that any samples in the dataset 
have been misidentified which could be problematic for other analyses.

% Coalescent
To account for incomplete lineage sorting, I also inferred phylogenetic     
relationships among \anaxyrus species using the multispecies coalescent 
method \phycoeval. 
\phycoeval was run with a subset of individuals used for the maximum likelihood 
tree due to increased computational demands of multispecies coalescent methods. 
The topology of the \phycoeval tree is substantially different from the  
maximum likelihood tree inferred in this study as well as trees from previous 
studies \cref{fig:phycoeval} \parencite{fontenot2011,graybeal1997,masta2002,pramuk2007,pyron2011,portik2023}. 
Unlike in any previous study or in the maximum likelihood tree, \amer and \terr are 
placed sister to one another, whereas in all other trees, \amer has had closer 
affinity to the \hemiophrys/\baxteri clade \cref{fig:iqtree-collapsed} \parencite{pyron2011,portik2023}.
In the \phycoeval tree, the \hemiophrys/\baxteri clade is instead sister
to the \amer/\fowl/\terr/\wood clade.
The placement of \fowl and \wood sister to one another by \phycoeval is also  
unlike any previous study or the maximum likelihood tree.

An unusual feature of \phycoeval is that it can allow for multifurcations in  
inferred topologies \parencite{oaks2022}.
This feature proved relevant in this study as the inferred tree included
one multifurcation at the ancestral node of \quercicus, the \cognatus/\speciosus
clade, and the \americanus group. 
Previous studies have produced trees with quite short internode branches at this
part of the tree as did the \iqtree analysis in this study. 
Most phylogenetic methods can only produce bifurcations and thus would force any true 
multifurcation into bifurcations and then have to estimate some branch length between nodes. 
In the \phycoeval tree, the posterior probability of this split is low (0.51) so 
may not be a perfect representation of the history of these lineages \cref{fig:phycoeval}.
More data may be necessary to have full resolution in this part of the tree.
But it is clear that these three lineages diverged at least in very rapid succession
if not simultaneously. 

\subsection{Divergence Time}
Only three previous studies have produced estimates for the age of \anaxyrus
or any of it's members \cite{frazao2015,feng2017,portik2023}.
The \cite{frazao2015} phylogeny places \incilius sister to \rhinella rather 
than \incilius which is not supported by most recent studies making 
their approximately 23 mya estimate for the origin of the genus questionable \cite{feng2017,portik2023,pyron2011}. 
\cite{portik2023} estimate the split between \anaxyrus and \incilius to be
20.3 mya (95\% HPD: 17.8-22.5) whereas \cite{feng2017} estimate a much earlier 
age of 16.5 mya (95\% CI: 14.0-19.4).
The dataset from \cite{feng2017} included near complete coverage from 95 nuclear loci 
whereas the \cite{portik2023} has a higher degree of missing data (95\%)
and includes both mitochondrial as well as nuclear loci. 
For these reasons I consider the \cite{feng2017} estimate to be the most reliable
and chose it for the rescaling the branch lengths of the \phycoeval tree.

Scaling the root of the \phycoeval tree with the \cite{feng2017} estimate
puts the time since the most recent common ancestor (MRCA) of extant \anaxyrus 
or the split between the \boreas group and rest of \anaxyrus,  
some time between 11.9 mya when \punctatus diverged from other \anaxyrus and 16.5 mya
when \anaxyrus split form \incilius \cref{fig:phycoeval}.
This range is consistent with the 12.3 mya estimate (95\% CI: 9.7-15.2) 
made by \cite{feng2017}.
But it would suggests that it must have happened almost immediately before the 
split leading to \punctatus. 
\cite{portik2023} estimate the age of the MRCA of all \anaxyrus to be approximately halfway 
between their 14.7 mya estimate for the \punctatus split and the 20.3 mya 
esimtate for the split with \incilius at 16.7 mya meaning. 
So not immediately preceding the split between \punctatus and all other \anaxyrus.
One thing this study and others have in common is a much higher degree of 
uncertainty around the ages of these basal splits in the \anaxyrus tree.
But they are all in support of the split between \incilius and \anaxyrus happening 
somewhere around the start or just before the middle of the Miocene epoch.
The MRCA of \anaxyrus and the split between the \boreas group with a Western 
distribution, likely occurred prior to the middle of the Miocene. 
My estimate for the split between \punctatus and other \anaxyrus would be 
right at the middle of the Miocene at a time when both precipitation and 
temperature underwent a decline in the North American interior which coincided
with an expansion of grasslands \parencite{morales-garcia2020}.
The timing of the multifurcation of the \quercicus, \cognatus/\speciosus, and 
\americanus group lineages coincides with a previously identified shift in the
ecomorphology of ungulate mammals inhabiting North America \parencite{morales-garcia2020}. 

I estimate that diversification of the \americanus group has all happened in the 
past 2.5 million years. This accounts for a large portion of the diversity of  
\anaxyrus and including \houstonensis which was not included in this study and others
have found to be nested within this clade \cite{portik2023,pyron2011}. 
This means that a large amount of diversification within \anaxyrus took place just before  
and during the Pleistocene 2.58 million to 11,700 years ago. 
This is a period marked by extreme climatic variation and repeated glacial cycles 
that transformed the climate and geography of the North American continent \parencite{holman1995,holman2003}.
Surprisingly, there is no evidence from the \phycoeval analysis that any single 
one of these cycles was a driver of multiple diversification events and instead 
that each event occurred independently during this period of \anaxyrus evolution.

\subsection{Hybridization}
There are numerous reports of hybridization among many different pairs of 
\anaxyrus species. 
However, the consequences of this hybridization are largely unknown. 
Using the \fbranch test, I found support for a modest level of introgression 
among several species pairs which have been previously reported to hybridize. 
Most of which presently exist in sympatry with one another.
The highest \fbranch statistics were calculated between \amer and \hemiophrys and 
between \amer and \baxteri with values of 0.24 and 0.22 respectively \cref{fig:dsuite}.
A hybrid zone is known to exist between \amer and \hemiophrys.
\cite{green1983} reported clinal variation of allozyme alleles at five different 
loci across an approximately 100 km transect in southeastern Manitoba, Canada.
The steep cline observed by \cite{green1983} over a relatively short distance suggests that reproductive 
isolation between these species is quite high. 
It is possible that introgression is occurring beyond this narrow hybrid zone
but the sample that I included here was sampled from a location in close proximity 
to the range of \amer \cref{fig:americanus-map} \parencite{conant1998}.
Thus, it is difficult to say if they detected introgression is shared by 
\hemiophrys as a whole or is only present within a hybrid zone.
Interestingly, there is also a high \fbranch score between \amer and \baxteri
which do no have ranges that are close to on another. 
It is possible that introgression from \amer occurred before the divergence  
of \hemiophrys and \baxteri and this is why both have high \fbranch scores. 
$f$-branch values can actually be indicative of introgression occurring between 
more basal branches of a tree in some circumstances \parencite{malinsky2021}.
This scenario is plausible as \baxteri is believed to be a relict of a more 
southerly distribution of \hemiophrys during a recent Pleistocene glacial period 
\parencite{henrich1968}.
Unfortunately, it is not possible to directly test for this scenario with \dsuite
due to limitations of the \fbranch test and without wider sampling from the range 
of \hemiophrys it is not possible to rule out recent introgression \parencite{malinsky2021}.

Several other \fbranch tests returned non-zero values although these were much
lower.
More than half of the \wood \fbranch statistics were greater than zero \amer \cref{fig:dsuite}.
Hybridization between all of these species is plausible as \wood occurs in 
sympatry at some part of its range with nearly all of them. 
Contemporary hybridization involving \wood has been reported with \amer, 
\cognatus, \microscaphus, and \speciosus \parencite{sullivan1986}. 
There is presently little to no overlap between \wood and \hemiophrys however
there could have been in the recent past due to Pleistocene glaciation  
pushing the range of \hemiophrys further south \parencite{henrich1968}. 
The two non-zero \fbranch values for \quercicus with \punctatus and \speciosus 
are perplexing.  
The distribution of \quercicus is confined to the pine woodlands of the 
Southeastern United States whereas the other two species are found in the  
short arid grasslands and deserts of the Southwest \parencite{conant1998}. 
The \fbranch statistic for the comparison between \punctatus and the common 
ancestor of \speciosus and \cognatus is more plausible given their broadly
overlapping distributions in the present day \parencite{conant1998}.

Unfortunately, there were many comparisons among \anaxyrus that could not be 
made using the \fbranch test due to the input tree topology. 
Particularly among ancestral species, between which past introgression could
be driving the pattern seen across the extant diversity as this is a known 
caveat with the \fbranch test \parencite{malinsky2021}.
The results presented here are consistent with introgression being an 
important factor in the evolutionary history of \anaxyrus but gaps remain.

The d-statistic class of methods for detecting introgression are not able to   
test for introgression between sister species so could not shed any light 
on putative hybridization between \fowl and \wood \parencite{meacham1962}. 
In order to test for admixture between \fowl and \wood I used the program 
\structure along with PCA.  
The results of both \structure and the PCA are consistent with the existence of  
a hybrid zone between these two species \cref{fig:structure-fowleri-woodhousii}.
Two \wood samples, one from Arkansas and the other from Texas, have large proportions 
of inferred ancestry from \fowl \cref{fig:structure-fowleri-woodhousii}. 
Several \fowl samples have large admixture proportions from \wood as well.  
The transition of ancestry proportions forms a steady East West gradient with 
one outlier present in Louisiana \cref{fig:structure-fowleri-woodhousii}.
The PCA results largely corroborate the results of the \structure analysis with 
\wood samples clustered tightly together, most \fowl samples clustering tightly
with a few deviating toward the center of the first principal compoenent axis, 
and finally two samples right in the center of the first principal component axis
\cref{fig:structure-fowleri-woodhousii}.
These results suggest the hybrid zone between \fowl and \wood is quite wide, 
possible on the order of hundreds of kilometers \cref{fig:structure-fowleri-woodhousii}.

This brings the number of confirmed \anaxyrus hybrid zones to three along with 
the \amer/\terr and \amer/\hemiophrys hybrid zones. 
Based on the \phycoeval phylogeny, all of these species emerged within the past 
2.5 million years.
This important context sheds light on the tempo of diversification within \anaxyrus.
The sister species pairs \fowl/\wood and \amer/\terr diverged only 0.7 and 1.0 mya
respectively \cref{fig:phycoeval}.
Within this timeframe neither of these species pairs has evolved a degree of 
reproductive isolation and/or character displacement that permits them to exist 
in sympatry with one another. 
Introgression across these hybrid zones extends a surprisingly long distance.
Introgression across the hybrid zone of \amer and \hemiophrys, two species with 
much older divergence times, appears to be much more limited on the other hand \parencite{green1983}. 
Despite having more recent divergence times, \fowl occurs in sympatry across 
a large area with both \amer and \terr and \wood overlaps significantly with 
\amer \parencite{conant1998}.
This is likely possible due to a higher degree of reproductive isolation that 
has evolved between these species pairs in the form of differences in 
advertisement call and timing of reproduction \parencite{blair1974}. 
Why more recently diverged species exist in sympatry and have evolved 
pre-zygotic isolating mechanisms whereas a \amer and \hemiophrys with much 
greater time since divergence do not is interesting.
Perhaps they have recently come into secondary contact and have not had sufficient
time to evolve pre-zygotic barriers to reproduction. 
This would lend support to reinforcement being the driving force behind the evolution
of pre-zygotic isolation in these taxa. 

\subsection{Population Structure}
An examination of population structure can potentially provide clues about the 
environmental factors that have shaped the evolutionary history of a species as 
population divergence is an early stage along the speciation continuum \parencite{mallet2008}.
The geographic barriers that result in the reduction of gene flow within species
could be the same types of barriers that have resulted in past speciation events 
involving a species or its close relatives.
In my analysis of population structure in \amer, \fowl, \terr, and \wood,
none of the \structure analyses reveal any discrete populations within these species.
The most abrupt transitions in admixture coefficients are seen within \amer.
The \structure model with the highest $Delta$K in the \amer analysis was the model
with a K of two which separates \amer into western and eastern populations
\cref{fig:evanno-amer} \parencite{evanno2005}. 
However, I will focus the discussion of the \structure results for \amer on the 
analysis run with a K of three despite it producing a likelihood that was only 
marginally better \cref{fig:evanno-amer}.
The results from the model with a K of three correspond well with the results of 
the PCA, make sense in a geographic context, and still show a stark 
transition of admixture coefficients from East to West as seen in the analysis 
with a K of 2 \cref{fig:structure-americanus,fig:structure-americanus2}.
The \structure analysis for \amer reveals a fairly abrupt transition from East
to West beginning at the Mississippi River \cref{fig:structure-americanus}.
All samples West of the Mississippi River have an admixture coefficient of one for the 
Western cluster of samples.
This cluster of individuals has a range that loosely corresponds with a proposed
subspecies \textit{A. anaxyrus charlesmithi} which was said to exist in parts  
of Oklahoma, Arkansas, Missouri, and along the margins of some bordering states 
\parencite{bragg1954}.
East of the Mississippi River, admixture coefficients associated with this cluster
shrink with increasing distance from the river. 
In the Southeastern direction, the admixture coefficient associated with samples
at the most Southeastern extent increase to one along this axis.
In the Northeastern direction is a more gradual transition from samples with 
a fairly balanced proportion of admixture from all three clusters to samples  
with a mixture of Northeastern and Southeastern ancestry, to finally a single
sample with an admixture coefficient of one for the cluster of samples associated
with this direction.
Samples in Eastern part of the \amer range appear to only vary with distance 
from one another and do not have any patterns of variation that are associated 
with any geographic feature as they are in the West.

The \structure results for \fowl, \terr, and \wood show very little if any
differentiation within species.
\terr shows the greatest level of differentiation among these three with 
eastern samples having ancestry attributed to a population not represented 
in any of the western samples (\cref{fig:structure-terrestris}). 
It is difficult to interpret this result with the extent and the size of the 
current dataset but it suggests there may be a gradient of genetic variation
across the range of \terr much like was found in \amer. 
Broader sampling which includes more samples from a greater extent of the \terr
range may shed light on this.  

The PCA for \fowl shows one tightly clustered group with 4 samples that 
stand out from the rest (\cref{fig:structure-fowleri}).
This corresponds to the number of samples which were inferred as having 
a proportion of \wood ancestry in the combined \fowl and \wood \structure 
analysis. 
The same four samples inferred as having \wood ancestry have the highest 
proportion of ancestry from the secondary population in the \fowl \structure 
analysis (\cref{fig:structure-fowleri}).
Apart from these individuals, the admixture proportions inferred for \fowl 
samples are highly uniform across the range of \fowl (\cref{fig:structure-fowleri}).  

Previous studies of \wood have identified two distinct groups of \wood. 
\cite{shannon1955} described the subspecies \textit{A. woodhousii australis} 
on the basis of morphological differentiation.
This subspecies was said to be distributed across the southern parts of Arizona
and New Mexico \parencite{shannon1955}. 
\cite{masta2003} found two divergent clades of \wood in a phylogeny inferred 
from a single mitochondrial locus. 
Samples from one of these clades matched the distribution of the subspecies described 
by although the distribution of samples from the two inferred clades 
overlapped to some extent \cite{shannon1955,masta2003}.
Sampling of \wood in this study is not as comprehensive as for other species
so may not be adequate to detect population structure consistent with 
previous findings of differentiation. 
However, the sampling does include one sample from Southwest New Mexico 
and the \structure analysis does not differentiate it from other samples \cref{fig:americanus-map}.
There are two samples assigned to a different population however these are the 
same two samples found to be highly admixed with \fowl. 


%%%%%%%%%%%%%%%%%%%%%%%%%%%%%%%%%%%%%%%%%%%%%%%%%%%%%%%%%%%%%%%%%%%%%%%%%%%%%%%%
\subsection{Conclusion}

Presented the first genome-wide dataset investigating the evolutionary 
history of \anaxyrus.


Overall, this study provides valuable insights into the complex evolutionary 
history of Anaxyrus species, highlighting the role of hybridization, speciation, 
and population structure. It underscores the need for comprehensive sampling and 
rigorous analyses to better understand the dynamics of species relationships and 
diversification within this genus.

The evolutionary history of \anaxyrus inferred in this study has some important 
implications for understanding the contemporary hybridization in \anaxyrus


Given the great interest in hybridization within \anaxyrus and the promise of this 
group for furthering our understanding speciation, it is important to consider the 
implications of the evolutionary history inferred in this study.


