Studying speciation in evolutionary biology is of paramount importance as it 
provides a fundamental understanding of how biodiversity emerges and evolves 
over time. 
Speciation, the process by which new species arise from a common ancestor, holds 
the key to unraveling the intricate tapestry of life on Earth.
It not only sheds light on the mechanisms driving genetic and phenotypic 
diversity but also offers crucial insights into the adaptation and survival of 
species in response to changing environments. 
Moreover, the study of speciation is essential for comprehending the origins of 
complex ecosystems and ecological interactions, as different species play distinct 
roles in shaping their environments. Ultimately, delving into the intricacies of 
speciation is not just an academic pursuit; it is a foundational pillar in our 
quest to understand life's past, present, and future, with implications for 
conservation, agriculture, and our own place in the natural world.



% Toads as an appealing target for study of hybridization
\mycomment{Planning to move this to intro chapter.}
It is clear that significant levels of admixture occur within some Bufonidae 
hybrid zones which could yield insights into the evolution of reproductive incompatibility. 
There are a few qualities that make these hybrid zones particularly attractive for further investigation.
One of these qualities is the ease with which the primary behavioral isolating mechanisms, 
spawning period and advertisement call, can be measured and quantified in order 
to understand the strength of prezygotic mating barriers and possible patterns 
consistent with reinforcement \parencite{cocroft1995,blair1974,kennedy1962}.
It has also been show that they can be readily bred in captivity \parencite{blair1972}. 
Many species produce thousands of offspring which are externally fertilized  
making a variety of embryological observations or manipulations possible \parencite{blair1972}.
Breeding can be induced hormonally or performed in vitro, facilitating the 
planning and scheduling of experiments \parencite{trudeau2010}.
Unlike many of the organisms which have undergone intensive study in the context 
of speciation such as \textit{Drosophila}, \textit{Mus}, and \textit{Heliconius},
most Bufonidae have homomorphic sex chromosomes \parencite{blair1972}. 
This is an interesting contrast in light of the important roll of sex chromosomes
have in the evolution of reproductive incompatibility and the roll of heterogamety
in explaining evolutionary patterns such as Haldane's rule, faster male evolution,
and faster-X evolution \parencite{delph2016}. 
Furthermore, there is evidence of sex chromosome turnovers within Bufonidae
which presents an opportunity to study differences in the evolution of  
reproductive incompatibility among closely related species with different 
sex determination systems \parencite{dufresnes2020,stock2011}. 
All of these qualities along with the near global distribution of a large 642 species  
radiation present an excellent opportunity to further our understanding of the 
evolution of reproductive incompatibility \parencite{amphibiaweb2023}.



% Outline
% P1
Speciation is a major focus of evolutionary biology

% P2
