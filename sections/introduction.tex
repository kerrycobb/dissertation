
% Introduction
Speciation is the driving force behind the incredible diversity of life on Earth. 
By studying this process, we gain insights into the mechanisms that 
have shaped the natural world, enabling us to appreciate and understand this 
biodiversity.
A fundamental aspect of speciation is the evolution of reproductive isolation.
This phenomenon has puzzled and captivated evolutionary biologists since the 
field's inception \parencite{mallet2008}.
Reproductive isolation was first viewed as incompatible with evolution by  
natural selection as it was inconceivable how natural selection might produce 
such an outcome \parencite{mallet2008}.
It is now appreciated that reproductive isolation is in fact compatible with 
evolution by natural selection, but mysteries abound \parencite{coyne2004}.
Why do organisms form discrete clusters instead of existing on a continuum?
What evolutionary processes drive reproductive isolation?
Which genes are involved and what are their functions under normal circumstances?
What are the targets of selection that drive evolution of these genes? 
What is the role of gene flow in the speciation process?
With increasingly powerful tools at their disposal, biologists are directing  
more attention than ever before into seeking answers to these questions.

% Toads
Toads in the family Bufonidae have held a prominent place in the speciation 
literature \parencite{blair1972}.
They have a number of qualities that makes them attractive for furthering our
understanding of the speciation process. 
Among these qualities is the ease with which the primary behavioral isolating mechanisms, 
spawning period and advertisement call, can be measured and quantified in order 
to understand the strength of prezygotic mating barriers and possible patterns 
consistent with reinforcement \parencite{cocroft1995,blair1974,kennedy1962}.
Researchers in the past were also drawn by the ease with which they can be 
crossed in the laboratory \parencite{blair1972}.
Spawning can be induced hormonally or performed in vitro, facilitating the planning  
and execution of experiments \parencite{trudeau2010}.
The females of many species can produce thousands of offspring which are externally 
fertilized making a variety of embryological observations or manipulations possible \parencite{blair1972}.
Many species pairs have proven to be reproductively compatible through laboratory crosses \parencite{blair1972}. 
This has provided an opportunity to investigate the tempo of the evolution of 
reproductive incompatibility and to understand the importance of pre-mating 
barriers \parencite{sasa1998,malone2008,fontenot2011}.
Another attractive quality of Bufonidae is the existence of several known 
hybrid zones which provide useful opportunities for studying the evolution 
of reproductive isolation in a natural setting \parencite{green1996,vanriemsdijk2023}. 

Unlike many organisms which have been the subject of intensive study in the 
context of speciation, such as \textit{Drosophila}, \textit{Mus}, and \textit{Heliconius},
most Bufonidae have homomorphic sex chromosomes \parencite{blair1972}.
This is an interesting contrast in light of the apparent importance of sex chromosomes
in the evolution of reproductive incompatibility and 
the roll of heterogamety in explaining evolutionary patterns such as Haldane's 
rule, faster male evolution, and faster-X evolution \parencite{delph2016}. 
Furthermore, there is evidence of sex chromosome turnovers within Bufonidae
which presents an opportunity to study differences in the evolution of  
reproductive incompatibility among closely related species with different 
sex determination systems \parencite{dufresnes2020,stock2011}. 
All of these qualities, along with a near global distribution and large diversity 
of species (642 species; \cite{amphibiaweb2023}) and ecological niches, make Bufonidae 
an excellent group to further our understanding of the evolution of reproductive 
incompatibility.



% Hybridization usefulness
In the first chapter of my dissertation I investigate a putative hybrid zone between 
\textit{Anaxyrus americanus} and \textit{A. terrestris}, two species in 
the family Bufonidae. 
Hybrid zones are increasingly appreciated to be a widespread phenomenon in nature.
One that can have important evolutionary consequences when it comes to the
process of speciation \parencite{moran2021}.
Hybrid zones also present a valuable opportunity to investigate reproductive 
incompatibility \parencite{rieseberg1999}. 
The production of large numbers of recombinant offspring through multiple 
generations of backcrossing under natural conditions cannot be achieved
through captive breeding for most organisms.

The only prior evidence of hybridization between \amer and \terr comes from anecdotal reports
and a single study of morphological variation across the contact zone between
these two species in central Alabama \parencite{mount1975,weatherby1982}. 
The amount of introgression, if any, within this putative hybrid zone is unknown.
Using genome-wide data collected from a large sample across the hybrid zone, 
I characterize introgression between these two species for the first time.  
I find that introgression between them is extensive and I identify many 
candidate loci that may be involved in reproductive incompatibility.
This study can serve as a guide to future studies in this system which could 
leverage quantitative measures of prezygotic isolation such 
as calls and breeding period, reference genomes, or crossing experiments paired
with cutting edge genomic tools such as CRISPR-Cas9 to shed light on the process
of speciation. 
In combination with other toad hybrid zones, there is a great deal we could 
learn about whether there are recurrent patterns in toad speciation, with
their homomorphic sex chromosomes are a useful contrast against other taxa.

% Evolutionary history
Hybrid zones have great potential for furthering our understanding of speciation.
However, they provide only a snapshot in time. 
There is a great deal of historical context that is also important to understand. 
Questions such as, how long has it been since hybridizing species have diverged?
What are the environmental factors that drive divergence between them? 
And, what are the lasting consequences of hybridization?
In the second chapter of my dissertation, I investigate the evolutionary history of 
species within the genus \anaxyrus to attempt to answer these questions. 
To accomplish this, I infer the phylogenetic relationships among species in this genus
from genome-wide sequence data and estimate the divergence times for nodes in 
the phylogenetic tree. 
I also test for a history of admixture to understand the importance of 
historical introgression among ancestral species during the evolutionary history of the genus.

To try to understand what factors might play a role in driving divergence
between populations and potentially result in speciation, I also investigate 
population structure within several species.
The relationships and divergence-time estimates that I infer from the genomic 
data differ substantially from previous studies based on more limited data
and methods \parencite{fontenot2011,graybeal1997,masta2002,pramuk2007,pyron2011,portik2023}.
I also find evidence of previously unrecognized hybridization in the past and 
present between species of \anaxyrus which reinforces the increasingly appreciated 
recognition of gene flow as a common and important process during the 
diversification of organisms.
My analysis of population structure shows strong population differentiation
in one species which could be in the early stage of speciation. 
This chapter highlights the utility of \anaxyrus for understanding speciation
as there is potentially hybridization occurring at multiple stages of the 
speciation process. 

% Methods
The inferences made about the evolutionary process in the previous two chapters
rely heavily on a suite of computational methods developed for the task. 
All of these methods make assumptions about the processes that produce the 
data, i.e. DNA sequences, which are used as input. 
These are often simplifying assumptions made to achieve tractable models and 
computation, many of which are known to be violated.
Many simplifying assumptions are sure to be violated and we know this to be the case for many of them.
Other assumptions are made because they reflect our best set of beliefs about 
the evolutionary process or because work has not yet been done to incorporate 
additional complexity into the methods. 
Many of these are sure to be violated as well but are more challenging to
recognize.
Violations of assumptions may not be highly problematic \parencite{oaks2020},
but the impact of many violations have never been evaluated. 
In the third chapter of my dissertation, I investigate the impact of 
errors and biases that can arise through the collection and processing DNA sequence data.
I find that these violations have a modest impact on inference. 
When clustering reads to construct an alignment, it is necessary to set a minimum  
similarity threshold that will very likely exclude variants from the alignment. 
In simulated data, I found this type of data acquisition bias had the effect of
underestimating recent divergence times and underestimating 
all effective population sizes.
This may be relevant for some of the very recent divergence times estimated for 
the \anaxyrus phylogeny and suggests caution should be taken in interpreting these. 


% Conclusion
This dissertation greatly enhances our understanding of the evolutionary history in \anaxyrus
and adds to a large body of research into speciation in Bufonidae that has been 
amassed over the past 60 years.
This dissertation also provides important context for understanding this past work.
Divergence time estimates give us an understanding of the tempo of diversification and for the evolution
of reproductive incompatibility.  
They also give us some clues as to the drivers of diversification within \anaxyrus.
I demonstrate that hybridization and introgression are important processes in 
the evolutionary history of \anaxyrus and provide confirmation of gene flow across
two hybrid zones for the first time.
Further analysis of these hybrid zones has promise for shedding light on the process
of speciation generally.
I also demonstrate why caution is necessary when interpreting the results of 
evolutionary inferences as the methods we rely on a suite of assumptions that
may commonly be violated.
This study lays a foundation for further advances in our understanding 
the process of speciation by taking advantage of many attractive qualities
this system offers.

