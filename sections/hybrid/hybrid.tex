\section{Introduction}

Hybrid zones have drawn the attention of biologist because they can teach us
about the process of speciation and may also have important consequences in the
evolution of hybridizing populations.

Hybridization in Bufonid toads has been appreciated for a long time. 

Hybridization experiments in the laboratory have demonstrated a high degree 
of compatibility between many species pairs.

Numerous reports of hybridization have been reported.

Several hybrid zones have been studied to understand the patterns of introgression within them. (Give examples)

Bufo bufo x Bufo spinosus in France \parencite{vanriemsdijk2023}
Bufo siculus x Bufo balearicus in Italy \parencite{colliard2010}
Anaxyrus americanus x Anaxyrus hemiophrys in Canada \parencite{green1983}


Here, I investigate a hybrid zone between two species of North American toad.
The American toad (\textit{Anaxyrus americanus}) and the southern toad 
(\textit{Anaxyrus terrestris}). The ranges of these two species do not overlap
but meet at a long zone of contact in the Southern United States. This contact 
zone corresponds closely with a prominent physiographic feature known as the fall  
line which is the boundary that separates the coastal plain to the South from the 
Appalachian Highlands to the North \parencite{shankman2007}.
The two species have slight differences in male advertisement call and in 
morphological appearance. They differ somewhat in the timing of their spawn  
with some overlap in periods.



\section{Methods}
\subsection{Sampling and DNA Isolation}
I collected genetic samples from \textit{A. americanus} and \textit{A. terrestris}
by driving roads during rainy nights between 2017 and 2020
in an a region of central Alabama where hybridization has previously been
inferred from the presence of morphological intermediates \parencite{weatherby1982}. 
I euthanized individuals with immersion in buffered MS-222.
I removed liver and/or toes and preserved them in 100\% ethanol and fixed 
specimens with XXX M (ask David how he makes Formalin) Formalin solution.
Genetic samples and formalin fixed specimens were deposited in the Auburn Museum of Natural History.
Additional samples were also provided by museums (see Table 1).

I isolated DNA by lysing a small piece of liver or toe approximately 
the size of a grain of rice in 300 \uL\ of a solution of 10mM Tris-HCL, 10mM EDTA, 
1\% SDS (w/v), and nuclease free water along with 6 mg Proteinase K and 
incubating for 4-16 hours at 55\degree C.  
To purify the DNA and separate it from the lysis product, I mixed the lysis 
product with a 2X volume of SPRI bead solution containing 1 mM EDTA,  
10 mM Tris-HCl, 1 M NaCl, 0.275\% Tween-20 (v/v), 18\% PEG 8000 (w/v), 
2\% Sera-Mag SpeedBeads (GE Healthcare PN 65152105050250) (v/v), and nuclease free water.
I then incubated the samples at room temperature for 5 minutes, placed the 
beads on a magnetic rack, and discarded the supernatant once the beads had collected
on the side of the tube.  
I then performed two ethanol washes by adding 1 mL of 70\% ETOH to the beads
while still placed in the magnet stand and allowing it to stand for 5 minutes
before removing and discarding the ethanol. 
After removing all ethanol from the second wash, I removed the tube from the magnet 
stand and allowed the sample to dry for 1 minute before thoroughly mixing the beads with 100 \uL\ of 
TLE solution containing 10 mM Tris-HCL, 0.1 mm EDTA, and nuclease free water.
After allowing the bead mixture to stand at room temperature for 5 minutes I returned
the beads to the magnet stand, collected the TLE solution, and discarded the beads. 
I quantified DNA in the TLE solution with a Qubit
fluorometer (Life Technologies, USA) and diluted samples with additional TLE solution to 
bring the concentration to 20 ng/\uL.

\subsection{RADseq Library Preparation}
I prepared RADseq libraries using the 2RAD approach developed by \cite{bayona-vasquez2019}. 
On 96 well plates, I ligated 100 ng of sample DNA in 15 \uL\ of a solution with 
1X CutSmart Buffer (New England Biolabs, USA; NEB), 10 units of XbaI,
10 units of EcoRI, 0.33 \uM\ XbaI compatible adapter, 0.33 \uM\ EcoRI compatible adapter,
and nuclease free water with a 1 hour incubation at 37\degree C. 
I then immediately added 5 \uL\ of a solution with 1X Ligase Buffer (NEB),
0.75 mM ATP (NEB), 100 units DNA Ligase (NEB), and nuclease free water 
and incubated at 22\degree C for 20 min and 37\degree C for 10 min for two cycles, 
followed by 80\degree C for 20 min to stop enzyme activity.
For each 96 well plate, I pooled 10 \uL\ of each sample and split this pool 
equally between two microcentrifuge tubes.
I purified each pool of libraries with a 1X volume of SpeedBead solution followed 
by two ethanol washes as described in the previous section except that the DNA 
was resuspended in 25 \uL\ of TLE solution and combined the two pools of cleaned 
ligation product. 

In order to be able to detect and remove PCR duplicates, I performed a single   
cycle of PCR with the iTru5-8N primer which adds a random 8 nucleotide barcode to 
each library construct.  
For each plate, I prepared four PCR reactions with a total volume of 
50 \uL\ containing 1X Kapa Hifi Buffer (Kapa Biosystems, USA; Kapa),
0.3 \uM\ iTru5-8N Primer, 0.3 mM dNTP, 1 unit Kapa HiFi DNA Polymerase,
10 \uL\ of purified ligation product, and nuclease free water.
I ran reactions through a single cycle of PCR on a thermocycler at 98\degree C for 2 min, 
60\degree C for 30 s, and 72\degree C for 5 min. 
I pooled all of the PCR products for a plate into a single tube and purified the
libraries with a 2X volume of SpeedBead solution as described before and 
resuspended in 25 \uL\ TLE.
I added the remaining adapter and index sequences which were unique to each plate with four PCR
reactions with a total volume of 50 \uL\ containing 1X Kapa Hifi (Kapa),
0.3 \uM\ iTru7 Primer, 0.3 \uM\ P5 Primer, 0.3 mM dNTP, 1 unit of Kapa Hifi DNA Polymerase (Kapa),
10 \uL\ purified iTru5-8N PCR product, and nuclease free water.
I ran reactions on a thermocycler with an initial denaturation at 98\degree C for 2 min, 
followed by 6 cycles of 98\degree C for 20 s, 60\degree C for 15 s, 72\degree C 
for 30 s and a final extension of 72\degree C for 5 min.
I pooled all of the PCR products for a plate into a single tube and purified the
product with a 2X volume of SpeedBead solution as described before and 
resuspended in 45 \uL\ TLE.

I size selected the library DNA from each plate in the range of 450-650 base pairs using
a BluePippin (Sage Science, USA) with a 1.5\% dye free gel with internal R2 standards. 
To increase the final DNA concentrations I prepared four PCR reactions for each 
plate with 1X Kapa Hifi (Kapa), 0.3 \uM\ P5 Primer, 0.3 \uM\ P7 Primer, 0.3 mM dNTP, 
1 unit of Kapa HiFi DNA Polymerase (Kapa), 10 \uL\ size selected DNA, and 
nuclease free water and used the same thermocycling conditions as the previous
(P5-iTru7) amplification.
I pooled all of the PCR products for a plate into a single tube and purified 
the product with a 2X volume of SpeedBead solution as before and resuspended in 20 \uL\ TLE. 
I quantified the DNA concentration for each plate with a Qubit fluorometer 
(Life Technologies, USA) then pooled each plate in equimolar amounts relative 
to the number of samples on the plate and diluted the pooled DNA to 5 nM with
TLE solution. 
The pooled libraries were pooled with other projects and sequenced on an Illumina 
HiSeqX by Novogene (China) to obtain paired end, 150 base pair sequences. 

\subsection{Data Processing}
I demultiplexed the iTru7 indexes using the \processradtags command from 
\stacks v2.6.4 \parencites{rochette2019} and allowed for two mismatches for rescuing reads.
To remove PCR duplicates, I used the \clonefilter command from \stacks.
I demultiplexed inline sample barcodes, trimmed adapter sequence, and filtered 
reads with low quality scores as well as reads with any uncalled bases using the  
\processradtags command again and allowed for the rescue of restriction site sequence 
as well as barcodes with up to two mismatches.  
I built alignments from the processed reads using the \stacks pipeline. 
I allowed for 14 mismatches between alleles within, as well as between individuals
(M and n parameters). This is equivalent to a sequence similarity threshold of   
90\% for the 140 bp length of reads post trimming. 
I also allowed for up to 7 gaps between alleles within and between individuals.
I used the \populations command from \stacks to filter loci missing in more than   
5\% of individuals, filter all sites with minor allele counts less than 3, filter 
any individuals with more than 90\% missing loci, and randomly sample a single
SNP from each locus.

\subsection{Genetic Clustering \& Ancestry Proportions}
To cluster individuals and characterize patterns of genetic differentiation and 
admixture between clusters, I used the Bayesian inference program 
\structure v2.3.4 \parencite{pritchard2000} with \structure's 
admixture model which returns an estimate of ancestry proportions for each sample. 
To evaluate the assumption that samples are best modeled as inheriting their    
genetic variation just two groups corresponding to the species identification  
made in the field, I ran \structure under four different models, each with a different number of 
assumed clusters of individuals (K parameter) ranging from 1 to 4. For each value of K, I 
ran 20 iterations for 100,000 total steps with the first 50,000 as burnin. 
I used the R package \pophelper v2.3.1 \parencite{francis2017} to combine
iterations for each value of K and to select the model producing the largest
$\Delta$K which is the the model that has the greatest increase in likelihood 
score from the previous model with K-1 as described by \parencite{evanno2005}.
I also examined genetic clustering and evidence of admixture using a non-parametric 
approach with a principal component analysis (PCA) implemented in the R package \adegenet v2.1.10 \parencite{jombart2008}. 
I visualized the relationship between the first principal component axis and the 
estimated admixture proportion for each individual to check for agreement  
between the parametric \structure analysis and the non-parametric PCA analysis.

\subsection{Genomic Cline Analysis}
To investigate patterns of introgression across the hybrid zone I     
used the bayesian genomic cline inference tool \bgc v1.03 \parencite{gompert2012} 
to infer parameters under a genomic cline model.
I classified a sample as being admixed if it had an inferred admixture proportion
of <95\% for one of the parent species under the model with a K of two in 
the \structure analysis.
I used \vcftools vXX.XX.XX to filter all non-biallelic sites from the the VCF file 
produced by the \populations program. 
I converted the VCF formatted data into the \bgc format using \bgcutils v0.1.0 
\url{https://github.com/kerrycobb/bgc_utils}.
I ran \bgc with 5 independent chains, each for 1,000,000 steps and sampling every 1000.
I discarded the first 10\% of samples from the posterior, combined the independent 
chains, summarized the posterior samples, and identified exceptional loci with \bgcutils. 

I classified loci as exceptional and therefore expected to be enriched for 
incompatibility loci with two different approaches, (1) if locus specific 
introgression differed from the genome-wide average and (2) if locus specific 
introgression is statistically unlikely relative to the genome-wide distribution
of locus specific introgression. 

More specifically, in the first approach I classified a locus as exceptional 
if the 90\% highest posterior density interval (HPDI) for the alpha or beta 
parameter did not cover zero.


% TODO: Combine the two paragraphs
In the second approach, I classified a locus as exceptional if the median of the  
posterior sample for the $\alpha$ or $\beta$ parameters for a locus were outside of 
the interval from 0.05 to 0.95 of the probability density functions $Normal(0, \tau_\alpha)$  
or $Normal(0, \tau_\beta)$ respectively, where $\tau_\alpha$ and $\tau_\beta$ are  
the median values from the posterior sample for the genome wide average 


the median of the posterior probability distribution was not contained within the interval from  
0.05 to 0.95 of the probability density function $Normal(0, \tau_\alpha)$ or 
$Normal(0, \tau_\beta)$ with $\tau_{\alpha}$ and $\tau_{\beta}$ being the medians of 
the posterior samples for conditional random effect priors on $\tau_\alpha$ 
and $\tau_\beta$ parameters as described by \parencite{gompert2011}.



\subsection{Genetic differentiation and Introgression}
I calculated \fst for each site used in the \bgc analysis using \vcftools  



\section{Results}
\subsection{Sampling and Data Processing}

I prepared reduced-representation sequencing libraries from 173 samples collected 
for this study (\cref{table:collectedHyb}) and 19 samples available from existing 
collections (\cref{table:loanedHyb})). After assembly and filtering, 43 samples
were excluded from analyses leaving a total of 149. 
Of the remaining samples, 56 had been identified 
as most closely resembling \amer and 93 had been identified as most closely
resembling \terr.
\stacks assembled reads into 432,336 loci with a mean length of 253.31 bp.
Prior to filtering the mean coverage per sample was 32X.
After filtering loci missing in greater than 5\% of samples, filtering sites with   
minor allele counts less than 3, filtering individuals with greater than 90\% 
missing loci, and randomly sampling a single SNP from each locus, 1194 sites
remained. 

A visual inspection of the \structure results shows that each iteration with  
same value for K converged on very similar results (\cref{fig:structureIter}). 

Using the method described by \parencite{evanno2005} indicates that a K of two
is the best model for \structure \cref{fig:deltak}.
Describe the approach...
Furthermore, individuals are inferred as having ancestry deriving from largely 
from two ancestral groups even for K values of three and four. \cref{fig:allk} 

Using 95\% ancestry proportion as a cutoff for considering individuals to have
pure ancestry, 

36 pure americanus
75 pure terrestris
38 admixed individuals

The admixed individuals are concentrated near the contact zone \cref{fig:hybrid-main} 
Some admixed individuals found quite far from the hybrid zone.


\section{Discussion}

% Effects of selection, idiosyncracies
With weak selection and low dispersal the false discovery rate will be higher \parencite{gompert2011}
Outlier loci are still expected to be enriched for genetic regions under selection.
False discovery rate when patterns of introgression are affected by epistasis
is quite high. Not all regions affecting introgression may be among outliers.  


% Multiple populations, potentially differing dynamics
One limitation of this study is samples included in analysis are sampled from 
a fairly broad area and may be more appropriately treated as separate populations.
They could be affected by different processes. %Cite study showing spatial and temporal differences across a hybrid zone
This hybrid zone could serve as an excellent system to explore such things
given the large extent. Many important environmental factors likely differ 
across the hybrid zone.


% Reference Genome
A reference genome for these species would greatly enhance the explanatory power of
this analysis allowing for an exploration of the genomic architecture of the   
reproductive isolation between these lineages. Which regions of the genome are 
involved in isolation. Are they more associated with sex chromosomes?
The availability of vertebrate genomes is increasing rapidly so it will likely 
be possible to map the outlier loci identified in this study to a reference genome
for \amer or \terr to reveal patterns in the distribution of these putative 
markers for genetic incompatibility. 


% Mitochondria
Mitochondrial data would also be valuable. Would allow for the detection of 
mitonuclear incompatibilities that could be contributing to reproductive isolation.
Could also be used to test for the effect of Haldane's rule which would predict  
that females would be more affected by hybridization. The estimated cline
for mitochondria would be expected to be much steeper \parencite{carling2008}.

% Geographic cline
With this study I have provided a characterization of this hybrid zone that can
serve as a guide for future sampling. 
Sampling along transects.
Could then apply geographic cline methods
Would give spatial insights.



